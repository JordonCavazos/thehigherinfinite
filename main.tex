%Jordon Cavazos
%Notes on "The Higher Infinite" 

\documentclass{article}
%\documentclass[a4paper, 11pt, oneside]{book} % A4 paper size, default 11pt font size and oneside for equal margins


%\usepackage[utf8]{inputenc} % Required for inputting international characters
%\usepackage[T1]{fontenc} % Output font encoding for international characters
%\usepackage{fouriernc} % Use the New Century Schoolbook font
\usepackage{amsthm}
\usepackage{mathtools}
\setcounter{section}{-1}


\newtheorem{theorem}{Theorem}[section]
\newtheorem{corollary}[theorem]{Corollary}
\newtheorem{lemma}[theorem]{Lemma}
\newtheorem{proposition}[theorem]{Proposition}

\theoremstyle{definition}
\newtheorem{definition}[theorem]{Definition}

\theoremstyle{remark}
\newtheorem*{remark}{Remark}



\begin{document} 

\begin{titlepage} % Suppresses headers and footers on the title page

	\centering % Centre everything on the title page
	
	\scshape % Use small caps for all text on the title page
	
	\vspace*{\baselineskip} % White space at the top of the page
	
	%------------------------------------------------
	%	Title
	%------------------------------------------------
	
	\rule{\textwidth}{1.6pt} \\ \vspace*{-\baselineskip}\vspace*{5pt} % Thick horizontal rule
	\rule{12cm}{1.4pt} \\ \vspace*{-\baselineskip}\vspace*{4.8pt}
	\rule{11cm}{1.2pt} \\ \vspace*{-\baselineskip}\vspace*{4.6pt}% Thin horizontal rule
	\rule{10cm}{1.0pt} \\ \vspace*{-\baselineskip}\vspace*{4.4pt}
	\rule{9cm}{0.8pt} \\ \vspace*{-\baselineskip}\vspace*{4.2pt}
	\rule{8cm}{0.6pt} \\ \vspace*{-\baselineskip}\vspace*{4.0pt}
	\rule{7cm}{0.4pt}%\vspace*{-\baselineskip}\vspace*{3.6pt}
	
%	\rule{7cm}{0.4pt} \\ \vspace*{-\baselineskip}\vspace*{4.2pt}
%	\rule{8cm}{0.6pt} \\ \vspace*{-\baselineskip}\vspace*{4.4pt}
%	\rule{9cm}{0.8pt} \\ \vspace*{-\baselineskip}\vspace*{4.6pt}
%	\rule{10cm}{1.0pt} \\ \vspace*{-\baselineskip}\vspace*{4.8pt}
%	\rule{11cm}{1.2pt} \\ \vspace*{-\baselineskip}\vspace*{5.0pt}
%	\rule{12cm}{1.4pt} \\ \vspace*{-\baselineskip}\vspace*{5.2pt}
%	\rule{\textwidth}{1.6pt} % Thick horizontal rule
	
	
	\vspace{\baselineskip} % Whitespace above the title
	
	{\LARGE GETTING TO KNOW THE CARDINALS\\} % Title
	
	\vspace{0.75\baselineskip} % Whitespace below the title
	
	%\rule{\textwidth}{1.6pt} \\ \vspace*{-\baselineskip}\vspace*{5pt} % Thick horizontal rule
	%\rule{12cm}{1.4pt} \\ \vspace*{-\baselineskip}\vspace*{4.8pt}
	%\rule{11cm}{1.2pt} \\ \vspace*{-\baselineskip}\vspace*{4.6pt}% Thin horizontal rule
	%\rule{10cm}{1.0pt} \\ \vspace*{-\baselineskip}\vspace*{4.4pt}
	%\rule{9cm}{0.8pt} \\ \vspace*{-\baselineskip}\vspace*{4.2pt}
	%\rule{8cm}{0.6pt} \\ \vspace*{-\baselineskip}\vspace*{4.0pt}
	%\rule{7cm}{0.4pt}%\vspace*{-\baselineskip}\vspace*{3.6pt}
	
	

	\rule{7cm}{0.4pt} \\ \vspace*{-\baselineskip}\vspace*{4.2pt}
	\rule{8cm}{0.6pt} \\ \vspace*{-\baselineskip}\vspace*{4.4pt}
	\rule{9cm}{0.8pt} \\ \vspace*{-\baselineskip}\vspace*{4.6pt}
	\rule{10cm}{1.0pt} \\ \vspace*{-\baselineskip}\vspace*{4.8pt}
	\rule{11cm}{1.2pt} \\ \vspace*{-\baselineskip}\vspace*{5.0pt}
	\rule{12cm}{1.4pt} \\ \vspace*{-\baselineskip}\vspace*{5.2pt}
	\rule{\textwidth}{1.6pt} % Thick horizontal rule
	
	\vspace{2\baselineskip} % Whitespace after the title block
	
	%------------------------------------------------
	%	Subtitle
	%------------------------------------------------
	
	Notes on Akihiro Kanamori's \emph{The Higher Inifite} % Subtitle or further description
	
	\vspace*{3\baselineskip} % Whitespace under the subtitle
	
	%------------------------------------------------
	%	Author
	%------------------------------------------------
	
	By
	
	\vspace{0.5\baselineskip} % Whitespace before the editors
	
	{\scshape\Large Jordon Cavazos\\} % Editor list
	
	\vspace{0.5\baselineskip} % Whitespace below the editor list
	
	
	\vfill % Whitespace between editor names and publisher logo
	
	%------------------------------------------------
	%	Publisher
	%------------------------------------------------
	
	
	\vspace{0.3\baselineskip} % Whitespace under the publisher logo
	 % Publication year
	

\end{titlepage}

\section{Preliminaries}
\begin{definition}

	For $X\subseteq \textrm{On}$, $\gamma$ is a \emph{limit point of $X$} iff $\bigcup (X \cap \gamma) = \gamma > 0$
\end{definition}
%\begin{remark}
	Notice that a limit point is necessarily a limit ordinal: if $\gamma = \alpha +1$ then $\bigcup \gamma = \alpha$ and we have $ \bigcup (X \cap \gamma) \subseteq \bigcup \gamma  = \alpha < \gamma$.
%\end{remark}
\begin{definition}
	$C$ is \emph{closed unbounded in $\delta$} iff $C$ is an unbounded subset of $\delta$ containing all its limit points less than $\delta$. 
\end{definition}
These sets are also called \emph{club} sets. Here $C$ is unbounded iff for all $x\in \delta$ there exists $y \in C$ such that $x<y$ (the alternative would have $x\leq y$). This definition of unbounded implies that $\delta$ must be infinite for club sets to exist.
\begin{definition}
	For regular $\nu < \delta$, $C$ is \emph{$\nu$-closed unbounded in $\delta$} iff $C$ is an unbounded subset of $\delta$ containing all its limit points less than $\delta$ of cofinality $\nu$. 
\end{definition}
\begin{definition}
	For limit ordinals $\delta$, $S$ is \emph{stationary in $\delta$} iff $S \subseteq \delta$ and $S\cap C \neq \emptyset$ for any $C$ closed unbounded in $\delta$. 
\end{definition}
\begin{definition}
	If $\langle X_\alpha \mid \alpha < \delta \rangle \in \prescript{\delta}{} {\mathcal{P}(\delta)}$, then its diagonal intersection is $\{ \xi < \delta \mid \xi \in \bigcap_{\alpha<\xi} X_\alpha \}$, denoted $\triangle_{\alpha < \delta} X_\alpha$.
\end{definition}

\begin{definition}
	For $X\subseteq \text{On}$ and $f: X \to \text{On}$, $f$ is \emph{regressive} iff $f(\alpha) < \alpha$ for every $\alpha \in X - \{\emptyset \}$.
\end{definition}

\begin{proposition}
	Suppose that $\lambda > \omega$ is regular. \par
	(a) If something \par
	(b) fkflkadsl;
\end{proposition}



\end{document}
