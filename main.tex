%Jordon Cavazos
%Notes on "The Higher Infinite" 

\documentclass{article}
%\documentclass[a4paper, 11pt, oneside]{book} % A4 paper size, default 11pt font size and oneside for equal margins


%\usepackage[utf8]{inputenc} % Required for inputting international characters
%\usepackage[T1]{fontenc} % Output font encoding for international characters
%\usepackage{fouriernc} % Use the New Century Schoolbook font
\usepackage{amsthm}
\usepackage{mathtools}
\setcounter{section}{-1}


\newtheorem{theorem}{Theorem}[section]
\newtheorem{corollary}[theorem]{Corollary}
\newtheorem{lemma}[theorem]{Lemma}
\newtheorem{proposition}[theorem]{Proposition}

\theoremstyle{definition}
\newtheorem{definition}[theorem]{Definition}

\theoremstyle{remark}
\newtheorem*{remark}{Remark}



\begin{document} 

\begin{titlepage} % Suppresses headers and footers on the title page

	\centering % Centre everything on the title page
	
	\scshape % Use small caps for all text on the title page
	
	\vspace*{\baselineskip} % White space at the top of the page
	
	%------------------------------------------------
	%	Title
	%------------------------------------------------
	
	\rule{\textwidth}{1.6pt} \\ \vspace*{-\baselineskip}\vspace*{5pt} % Thick horizontal rule
	\rule{12cm}{1.4pt} \\ \vspace*{-\baselineskip}\vspace*{4.8pt}
	\rule{11cm}{1.2pt} \\ \vspace*{-\baselineskip}\vspace*{4.6pt}% Thin horizontal rule
	\rule{10cm}{1.0pt} \\ \vspace*{-\baselineskip}\vspace*{4.4pt}
	\rule{9cm}{0.8pt} \\ \vspace*{-\baselineskip}\vspace*{4.2pt}
	\rule{8cm}{0.6pt} \\ \vspace*{-\baselineskip}\vspace*{4.0pt}
	\rule{7cm}{0.4pt}%\vspace*{-\baselineskip}\vspace*{3.6pt}
	
%	\rule{7cm}{0.4pt} \\ \vspace*{-\baselineskip}\vspace*{4.2pt}
%	\rule{8cm}{0.6pt} \\ \vspace*{-\baselineskip}\vspace*{4.4pt}
%	\rule{9cm}{0.8pt} \\ \vspace*{-\baselineskip}\vspace*{4.6pt}
%	\rule{10cm}{1.0pt} \\ \vspace*{-\baselineskip}\vspace*{4.8pt}
%	\rule{11cm}{1.2pt} \\ \vspace*{-\baselineskip}\vspace*{5.0pt}
%	\rule{12cm}{1.4pt} \\ \vspace*{-\baselineskip}\vspace*{5.2pt}
%	\rule{\textwidth}{1.6pt} % Thick horizontal rule
	
	
	\vspace{\baselineskip} % Whitespace above the title
	
	{\LARGE GETTING TO KNOW THE CARDINALS\\} % Title
	
	\vspace{0.75\baselineskip} % Whitespace below the title
	
	%\rule{\textwidth}{1.6pt} \\ \vspace*{-\baselineskip}\vspace*{5pt} % Thick horizontal rule
	%\rule{12cm}{1.4pt} \\ \vspace*{-\baselineskip}\vspace*{4.8pt}
	%\rule{11cm}{1.2pt} \\ \vspace*{-\baselineskip}\vspace*{4.6pt}% Thin horizontal rule
	%\rule{10cm}{1.0pt} \\ \vspace*{-\baselineskip}\vspace*{4.4pt}
	%\rule{9cm}{0.8pt} \\ \vspace*{-\baselineskip}\vspace*{4.2pt}
	%\rule{8cm}{0.6pt} \\ \vspace*{-\baselineskip}\vspace*{4.0pt}
	%\rule{7cm}{0.4pt}%\vspace*{-\baselineskip}\vspace*{3.6pt}
	
	

	\rule{7cm}{0.4pt} \\ \vspace*{-\baselineskip}\vspace*{4.2pt}
	\rule{8cm}{0.6pt} \\ \vspace*{-\baselineskip}\vspace*{4.4pt}
	\rule{9cm}{0.8pt} \\ \vspace*{-\baselineskip}\vspace*{4.6pt}
	\rule{10cm}{1.0pt} \\ \vspace*{-\baselineskip}\vspace*{4.8pt}
	\rule{11cm}{1.2pt} \\ \vspace*{-\baselineskip}\vspace*{5.0pt}
	\rule{12cm}{1.4pt} \\ \vspace*{-\baselineskip}\vspace*{5.2pt}
	\rule{\textwidth}{1.6pt} % Thick horizontal rule
	
	\vspace{2\baselineskip} % Whitespace after the title block
	
	%------------------------------------------------
	%	Subtitle
	%------------------------------------------------
	
	Notes on Akihiro Kanamori's \emph{The Higher Inifite} % Subtitle or further description
	
	\vspace*{3\baselineskip} % Whitespace under the subtitle
	
	%------------------------------------------------
	%	Author
	%------------------------------------------------
	
	By
	
	\vspace{0.5\baselineskip} % Whitespace before the editors
	
	{\scshape\Large Jordon Cavazos\\} % Editor list
	
	\vspace{0.5\baselineskip} % Whitespace below the editor list
	
	
	\vfill % Whitespace between editor names and publisher logo
	
	%------------------------------------------------
	%	Publisher
	%------------------------------------------------
	
	
	\vspace{0.3\baselineskip} % Whitespace under the publisher logo
	 % Publication year
	

\end{titlepage}

\section{Preliminaries}
\begin{definition}

	For $X\subseteq \textrm{On}$, $\gamma$ is a \emph{limit point of $X$} iff $\bigcup (X \cap \gamma) = \gamma > 0$
\end{definition}
%\begin{remark}
	Notice that a limit point is necessarily a limit ordinal: if $\gamma = \alpha +1$ then $\bigcup \gamma = \alpha$ and we have $ \bigcup (X \cap \gamma) \subseteq \bigcup \gamma  = \alpha < \gamma$.
%\end{remark}
\begin{definition}
	$C$ is \emph{closed unbounded in $\delta$} iff $C$ is an unbounded subset of $\delta$ containing all its limit points less than $\delta$. 
\end{definition}
These sets are also called \emph{club} sets. Here $C$ is unbounded in $\delta$ iff for all $x\in \delta$ there exists $y \in C$ such that $x\leq y$ (the alternative would have $x<y$). Equivalently, $\sup(C)=\sup(\delta)$. Also equivalently,  $C$ is cofinal in $\delta$.
\begin{definition}
	For regular $\nu < \delta$, $C$ is \emph{$\nu$-closed unbounded in $\delta$} iff $C$ is an unbounded subset of $\delta$ containing all its limit points less than $\delta$ of cofinality $\nu$. 
\end{definition}
\begin{definition}
	For limit ordinals $\delta$, $S$ is \emph{stationary in $\delta$} iff $S \subseteq \delta$ and $S\cap C \neq \emptyset$ for any $C$ closed unbounded in $\delta$. 
\end{definition}
\begin{definition}
	If $\langle X_\alpha \mid \alpha < \delta \rangle \in \prescript{\delta}{} {\mathcal{P}(\delta)}$, then its diagonal intersection is $\{ \xi < \delta \mid \xi \in \bigcap_{\alpha<\xi} X_\alpha \}$, denoted $\triangle_{\alpha < \delta} X_\alpha$.
\end{definition}

\begin{definition}
	For $X\subseteq \text{On}$ and $f: X \to \text{On}$, $f$ is \emph{regressive} iff $f(\alpha) < \alpha$ for every $\alpha \in X - \{\emptyset \}$.
\end{definition}

\begin{proposition}
	Suppose that $\lambda > \omega$ is regular. \par
	(a) If $\gamma <\lambda$ and $\langle C_\alpha \mid \alpha <\gamma \rangle$ is a sequence of sets closed unbounded in $\lambda$, then $\bigcap_{\alpha <\gamma} C_\alpha$ is closed unbounded in $\lambda$. \par
	(b) If  $\langle C_\alpha \mid \alpha <\lambda \rangle$ is a sequence of sets closed unbounded in $\lambda$, then its diagonal intersection $\triangle_{\alpha < \lambda}C_\alpha$ is closed unbounded in $\lambda$. \par
	(c) \emph{(Fodor)} If $S$ is stationary in $\lambda$ and $f:S\to\lambda$ is regressive, then there is an $\alpha<\lambda$ such that $f^{-1}(\{\alpha\})$ is stationary in $\lambda$. \par
	(d) If $\nu<\lambda$ is regular, $S\subseteq\{\xi<\lambda\mid\operatorname{cf}(\xi)=\nu\}$ is stationary in $\lambda$, and $C$ is $\nu$-closed unbounded in $\lambda$, then $S\cap C \neq\emptyset$.
\end{proposition}

\begin{proof}
(a) Let $C=\bigcap_{\alpha <\gamma} C_\alpha$. First, we show $C$ is closed. Notice for all sets $X$ and ordinals $\beta$, $\sup(\beta\cap X)\leq \beta$. Suppose $\sup(\delta\cap C)=\bigcup\delta\cap C=\delta$. Then for all $\alpha<\gamma$, $(\delta\cap C)\subseteq (\delta\cap C_\alpha)$. Thus $\delta=\sup(\delta\cap C)\leq\sup(\delta\cap C_\alpha)\leq \delta$. Hence $\sup(\delta\cap C_\alpha)=\delta$. Hence $\delta$ is a limit point and since each $C_\alpha$ is closed, $\delta \in C_\alpha$ for all $\alpha<\gamma$. Thus $\delta \in C$. Hence $C$ is closed. \par
Now we show $C$ is unbounded. Fix $\beta<\lambda$. Define $b_0=\beta$. Given $b_n<\lambda$ for $n<\omega$, for each $\alpha$ choose $b_{\alpha,n}\in C_\alpha$ such that $b_{\alpha,n} \geq b_n$. Define $b_{n+1}=\sup\{b_{\alpha,n} \mid \alpha<\gamma\}$. Since $\lambda$ is regular and $\gamma<\lambda$, $b_{n+1}<\lambda$. Thus, by unboundedness of each $C_\alpha$, we can define the sequence $\{b_n \mid n<\omega\}$. For each $\alpha$, $b_0\leq b_{\alpha,0}\leq b_1 \leq b_{\alpha,1} \leq \cdots$, thus $\delta=\sup\{b_n \mid n<\omega\}=\sup\{b_{\alpha,n} \mid n<\omega\}$. Furthermore, since $\lambda > \omega$ regular, $\delta < \lambda$. If $\delta = b_{\alpha,n}$ for some $n$ then for all $m+1>n$ and for all $\alpha < \gamma$, $\delta \leq b_{\alpha,m}\leq b_m \leq \delta$. Hence for all $\alpha$, $\delta \in C_\alpha$. Thus we have $\delta \in C$ with $\delta\geq\beta$. Otherwise, $\delta$ is a limit ordinal and, for all $\alpha$, we have $\delta=\sup(\delta\cap \{b_{\alpha,n}\mid n<\omega\})\leq\sup(\delta\cap C_\alpha)\leq\delta$. Hence $\delta$ is a limit point less than $\lambda$ for each $C_\alpha$ and, by closedness, $\delta \in C_\alpha$. Hence we have $\delta \in C$ with $\delta \geq \beta$. Thus $C$ is unbounded. Thus $C$ is closed unbounded in $\lambda$.  \par 
(b) 
\end{proof}


\end{document}
